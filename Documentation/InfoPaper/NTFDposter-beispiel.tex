\documentclass[portrait,final,a0paper,fontscale=0.277]{NTFDposter}

\usepackage{setspace}
\definecolor{Headerblue}{cmyk}{1,.30,0,0}
%%%%%%%%%%%%%%%%%%%%%%%%%%%%%%%%%%%%%%%%%%%%%%%%%%%%%%%%%%%%%%%%%%%%%%%%%%%%%%%%
%%%% Some math symbols used in the text
%%%%%%%%%%%%%%%%%%%%%%%%%%%%%%%%%%%%%%%%%%%%%%%%%%%%%%%%%%%%%%%%%%%%%%%%%%%%%%%%
\makeatletter
\renewcommand{\NTFDbausteine@farbschema}{rgb}
\renewcommand{\NTFDbausteine@farbtafel}{l}
\makeatother
%%%%%%%%%%%%%%%%%%%%%%%%%%%%%%%%%%%%%%%%%%%%%%%%%%%%%%%%%%%%%%%%%%%%%%%%%%%%%%%%%
%% Multicol Settings
%%%%%%%%%%%%%%%%%%%%%%%%%%%%%%%%%%%%%%%%%%%%%%%%%%%%%%%%%%%%%%%%%%%%%%%%%%%%%%%%%
\setlength{\columnsep}{1.5em}
\setlength{\columnseprule}{0mm}

%%%%%%%%%%%%%%%%%%%%%%%%%%%%%%%%%%%%%%%%%%%%%%%%%%%%%%%%%%%%%%%%%%%%%%%%%%%%%%%%%
%% Save space in lists. Use this after the opening of the list
%%%%%%%%%%%%%%%%%%%%%%%%%%%%%%%%%%%%%%%%%%%%%%%%%%%%%%%%%%%%%%%%%%%%%%%%%%%%%%%%%
%\newcommand{\compresslist}{%
%\setlength{\itemsep}{1pt}%
%\setlength{\parskip}{0pt}%
%\setlength{\parsep}{0pt}%
%}

%%%%%%%%%%%%%%%%%%%%%%%%%%%%%%%%%%%%%%%%%%%%%%%%%%%%%%%%%%%%%%%%%%%%%%%%%%%%%%
%%% Begin of Document
%%%%%%%%%%%%%%%%%%%%%%%%%%%%%%%%%%%%%%%%%%%%%%%%%%%%%%%%%%%%%%%%%%%%%%%%%%%%%%


  
\begin{document}

	\author[iec]{Max Mustermann\corref{cor1}\fnref{fn1}}
    \ead{cvr@river-valley.com}
    \ead[url]{ntfd.tu-freiberg.de}
    
    \author[vtc]{K.~Bazargan\corref{cor2}}
    \ead{cvr@river-valley.com}
     
     \author{Maxi Mustermann}
    \ead{cvr@river-valley.com}
    
    \author{K.~Bazargani}
    \ead{cvr@river-valley.com}
   
   \address[iec]{River Valley Technologies, SJP Building, Cotton Hills, Trivandrum, Kerala, India 695014}
   \address[vtc]{Institut f\"ur Energieverfahrenstechnik und Chemieingenieurwesn, Fuchsm\"uhlenweg 9, 09599 Freiberg}
   
       \cortext[cor1]{Corresponding author}
       \cortext[cor2]{Corresponding author 2}
    \fntext[fn1]{This is the specimen author footnote.}
   
   \title{Optimal Landmark Detection using Shape Models and Branch and Bound }
%%%%%%%%%%%%%%%%%%%%%%%%%%%%%%%%%%%%%%%%%%%%%%%%%%%%%%%%%%%%%%%%%%%%%%%%%%%%%%
%%% Here starts the poster
%%%---------------------------------------------------------------------------
%%% Format it to your taste with the options
%%%%%%%%%%%%%%%%%%%%%%%%%%%%%%%%%%%%%%%%%%%%%%%%%%%%%%%%%%%%%%%%%%%%%%%%%%%%%% 

\hyphenation{resolution occlusions}
%%

\begin{poster}%
  % Poster Options
  {
  % Show grid to help with alignment
  grid=false,
  % Column spacing
  colspacing=1em,
  % Color style
  bgColorOne=white,
  bgColorTwo=white,
  borderColor=black,
  headerColorOne=TUBAFhausfarbe,
  headerColorTwo=TUBAFhausfarbe,
  headerFontColor=white,
  boxColorOne=white,
  boxColorTwo=white,
  % Format of textbox
  textborder=rounded,
  % Format of text header
  eyecatcher=false,
  headerborder=closed,
  headerheight=0.3\textheight,
%  textfont=\sc, An example of changing the text font
  headershape=rounded,
  headershade=plain,
  headerfont=\Large\bf\textsc, %Sans Serif
  textfont={\setlength{\parindent}{1.5em}},
  boxshade=plain,
%  background=shade-tb,
  background=plain,
  linewidth=1pt
  }
  % Title
  {}
  % Authors
  {}
 
%%%%%%%%%%%%%%%%%%%%%%%%%%%%%%%%%%%%%%%%%%%%%%%%%%%%%%%%%%%%%%%%%%%%%%%%%%%%%%
%%% Now define the boxes that make up the poster
%%%---------------------------------------------------------------------------
%%% Each box has a name and can be placed absolutely or relatively.
%%% The only inconvenience is that you can only specify a relative position 
%%% towards an already declared box. So if you have a box attached to the 
%%% bottom, one to the top and a third one which should be in between, you 
%%% have to specify the top and bottom boxes before you specify the middle 
%%% box.
%%%%%%%%%%%%%%%%%%%%%%%%%%%%%%%%%%%%%%%%%%%%%%%%%%%%%%%%%%%%%%%%%%%%%%%%%%%%%%
    %
    % A coloured circle useful as a bullet with an adjustably strong filling
    \newcommand{\colouredcircle}{%
      \tikz{\useasboundingbox (-0.2em,-0.32em) rectangle(0.2em,0.32em); \draw[draw=black,fill=lightblue,line width=0.03em] (0,0) circle(0.18em);}}

%%%%%%%%%%%%%%%%%%%%%%%%%%%%%%%%%%%%%%%%%%%%%%%%%%%%%%%%%%%%%%%%%%%%%%%%%%%%%%
  \headerbox{Problem}{name=problem1,column=0,row=0}{
%%%%%%%%%%%%%%%%%%%%%%%%%%%%%%%%%%%%%%%%%%%%%%%%%%%%%%%%%%%%%%%%%%%%%%%%%%%%%%
  Fitting statistical 2D and 3D shape models to images is necessary for a
  variety of tasks, such as video editing and face recognition. Much progress
  has been made on local fitting from an initial guess, but determining a close
  enough initial guess is still an open problem. We propose a method to locate
  fiducial points, which can then be used to initialize the fitting.
 }
 %%%%%%%%%%%%%%%%%%%%%%%%%%%%%%%%%%%%%%%%%%%%%%%%%%%%%%%%%%%%%%%%%%%%%%%%%%%%%%
  \headerbox{Problem}{name=problem2,column=1,span=2,row=0}{
%%%%%%%%%%%%%%%%%%%%%%%%%%%%%%%%%%%%%%%%%%%%%%%%%%%%%%%%%%%%%%%%%%%%%%%%%%%%%%
  Fitting statistical 2D and 3D shape models to images is necessary for a
  variety of tasks, such as video editing and face recognition. Much progress
  has been made on local fitting from an initial guess, but determining a close
  enough initial guess is still an open problem. We propose a method to locate
  fiducial points, which can then be used to initialize the fitting.
    Fitting statistical 2D and 3D shape models to images is necessary for a
  variety of tasks, such as video editing and face recognition. Much progress
  has been made on local fitting from an initial guess, but determining a close
  enough initial guess is still an open problem. We propose a method to locate
  fiducial points, which can then be used to initialize the fitting.
  Fitting statistical 2D and 3D shape models to images is necessary for a
  variety of tasks, such as video editing and face recognition. Much progress
  has been made on local fitting from an initial guess, but determining a close
  enough initial guess is still an open problem. We propose a method to locate
  fiducial points, which can then be used to initialize the fitting.

 }
%%%%%%%%%%%%%%%%%%%%%%%%%%%%%%%%%%%%%%%%%%%%%%%%%%%%%%%%%%%%%%%%%%%%%%%%%%%%%%
  \headerbox{Problem}{name=problem3,column=0,below=problem1}{
%%%%%%%%%%%%%%%%%%%%%%%%%%%%%%%%%%%%%%%%%%%%%%%%%%%%%%%%%%%%%%%%%%%%%%%%%%%%%%
  Fitting statistical 2D and 3D shape models to images is necessary for a
  variety of tasks, such as video editing and face recognition. Much progress
  has been made on local fitting from an initial guess, but determining a close
  enough initial guess is still an open problem. We propose a method to locate
  fiducial points, which can then be used to initialize the fitting.
 }
 %%%%%%%%%%%%%%%%%%%%%%%%%%%%%%%%%%%%%%%%%%%%%%%%%%%%%%%%%%%%%%%%%%%%%%%%%%%%%%
  \headerbox{Problem}{name=problem3,column=2,below=problem2}{
%%%%%%%%%%%%%%%%%%%%%%%%%%%%%%%%%%%%%%%%%%%%%%%%%%%%%%%%%%%%%%%%%%%%%%%%%%%%%%
  Fitting statistical 2D and 3D shape models to images is necessary for a
  variety of tasks, such as video editing and face recognition. Much progress
  has been made on local fitting from an initial guess, but determining a close
  enough initial guess is still an open problem. We propose a method to locate
  fiducial points, which can then be used to initialize the fitting.
 }
\end{poster}

\end{document}

