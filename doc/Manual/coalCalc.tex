%        File: volatilization.tex
%     Created: Mon Jul 11 10:00 AM 2011 C
% Last Change: Mon Jul 11 10:00 AM 2011 C
%
%\documentclass[a4paper]{article}
\documentclass[a4paper]{scrartcl}
\usepackage{amssymb}
\usepackage[final,colorlinks=true,hidelinks=true]{hyperref}

%\usepackage[collision]{chemsym}
\usepackage[]{bpchem}
%\usepackage[mediumspace,mediumqspace,squaren]{SIunits}
\usepackage{siunitx}
\usepackage{chemarrow}
\usepackage{subfigure}
%% MACRO
%% Chemical compounds
\usepackage[version=3]{mhchem}
%\newcommand{\hy}{$H_2$}		% hydrogen H2
\newcommand{\hy}{\ce{H2}}		% hydrogen H2
\newcommand{\ox}{\ce{O2}}		% oxygen O2
\newcommand{\cd}{\ce{CO2}}		% carbon dioxide CO2
\newcommand{\cm}{\ce{CO}}		% carbon dioxide CO
\newcommand{\hs}{\ce{H2S}}		% sulphur hydride H2S
\newcommand{\wat}{\ce{H2O}}	% water
\newcommand{\met}{\ce{CH4}}	% CH4
\newcommand{\NOx}{\ce{NO_x}}	% NOx
\newcommand{\ny}{\ce{N2}}		% N2
\newcommand{\NO}{\ce{NO}}		% NO
\newcommand{\fluent}{Fluent\texttrademark\xspace}
\newcommand{\icem}{IcemCFD\texttrademark\xspace}

\newcommand{\half}{\frac{1}{2}}
%%%%%
% prefessional table
\usepackage{booktabs}
%% Packages to draw nicely tables
%\usepackage{ucs}                                         
\usepackage{color}                                     
\usepackage{array}                                        
\usepackage{longtable}                                     
\usepackage{calc}                                           
\usepackage{multirow}                                        
\usepackage{hhline}                                           
\usepackage{ifthen}                                           
%%  optionally (for landscape tables embedded in another document): %%
\usepackage{lscape}                                           
\usepackage{rotating}
\usepackage{url}
\usepackage{tabularx}
% end packages
% reccomanded see L'arte di gestire bibliografia
\usepackage[babel]{csquotes}
\usepackage[style = numeric,
	citestyle = numeric-comp,
	sorting = none,
	isbn = false,
	url = false,
	doi = false,
	maxcitenames=2,
	mincitenames=1
	]{biblatex}
%\addbibresource{papers.bib}
\bibliography{/home/vascella/Documents/papers/papers}

\usepackage{pgfplots}
\pgfplotsset{compat=1.3}

\title{Calculation of coal properties}
\author{Michele Vascellari}

%\usepackage{times}
\begin{document}
\maketitle

\section{Introduction}

\section{CPD model}

\subsection{Volatile composition} % (fold)
\label{sub:Volatile}

CPD provides as input the following species composition as volatile matter: char, tar, \met, \cd,
\cm, \wat\ and one generic species defined as \emph{others}.

The composition is defined interpolating between experimentally defined coal.

Considering as input the ultimate analysis in terms of daf fractions of C, H, O and N and the
volatile yield from CPD $f_i$ on daf, for the previously defined species (char, tar, \met, \cd,
\cm, \wat and others), the following steps are required to define a volatile yield that satisfy the
element balance.

\subsubsection{Oxygen check}

First of all, the overall oxygen contents of the light gas species\footnote{tar and char are not
considered here} on the volatile must be lower than the oxygen from the ultimate analysis:

\begin{eqnarray}
	y_\ce{O}^{cpd} = M_\ce{O} \sum_i \frac{\mu_i^\ce{O} f_i}{M_i} \le y_\ce{O} 
	\label{eq:O_balance}
\end{eqnarray}

Where $i$ can be \cm, \cd\ and \wat. $\mu_i$ is the number of atoms of O in the
molecule\footnote{$\mu_\cd=2$, $\mu_\cm=1$ and $\mu_\wat=1$}. $M_i$ and $f_i$ are respectively the
molecular weight and the mass fraction on daf of species $i$.

If Eq.~\eqref{eq:O_balance} is not verified, it is necessary to scale up the species containing
oxygen in order to satisfy this constrain. \cd, \cm\ and \wat\ must be scaled up of the
factor:

\begin{eqnarray}
	\gamma = \frac{y_\ce{O}}{y_\ce{O}^{cpd}}
	\label{eq:scale1}\\
	f_i^{new} = \gamma \cdot f_i 
	\label{eq:scale2}
\end{eqnarray}

The removed part of volatile matter removed must be added to the others species\footnote{Note that
$f_i$ is referred to the old volatile matter mass fraction}:

\begin{eqnarray}
	f_{oth}^{new} = f_{oth} + \left(1-\gamma\right) \sum_i f_i
	\label{eq:add}
\end{eqnarray}

Once that oxygen balance was checked, it is necessary to assign the contents of the ``others''. Part
of it is assigned to nitrogen in order to fulfill its balance:

\begin{eqnarray}
	f_\ny = y_\ce{N}
	\label{eq:nitro}
\end{eqnarray}

If $f_\ny < f_{oth}$ the remaining part should be assigned to \met\footnote{It could be assigned to
\hy, but generally too high C:H ratios are obtained for tar}:

\begin{eqnarray}
	f_\met^{new} = f_\met + \left( f_{oth}^{new} - f_\ny \right)
	\label{eq:ch4}
\end{eqnarray}

When the composition of volatile is completely defined, the tar composition can be calculated. The
following molecule is considered \ce{C_n H_m O_p}. Once defined an arbitrary molecular weight to the
tar $M_{tar}$, the balance for each element can be written:

\begin{eqnarray}
	\frac{y_j}{M_j} = \mu_{tar}^j \frac{f_{tar}}{M_{tar}} + \sum_i \mu_i^{j} \frac{f_{i}}{M_{i}}
	\label{eq:tar}
\end{eqnarray}

Where $y_j$ and $M_j$ are the mass fraction from ultimate analysis and atom weight of C, H, O.
$\mu_i^j$ is the number of atom $j$ in the species $i$ \footnote{Now the new mass fraction
$f_i$ are considered}. From Eq.~\eqref{eq:tar} is it possible to
find the composition of tar $\mu_{tar}^j$, given the molecular weight $M_{tar}$.

% subsection Volatile composition (end)


\section{Energy balance}

The high heating value of coal as received, can be related to the daf:

\begin{eqnarray}
	HHV_{ar} = y_{daf} \cdot HHV_{daf}
	\label{eq:hhv}
\end{eqnarray}

where $y_{daf}$ is the dry ash free fraction of coal, calculated considering coal as received.
The lower heating value LHV is:

\begin{eqnarray}
	LHV_{daf} = HHV_{daf} - \frac{M_\wat}{2 M_\ce{H}}\cdot y_\ce{H} \cdot r_\wat
	\label{eq:lhv}
\end{eqnarray}

Where $r_\wat$ is the latent heat of \wat.

Considering a raw coal species \ce{C_x H_y O_z N_w} of given molecular weight $M_{raw}$, its
enthalpy of formation can be calculated considering the energy balance for a generic reaction at
reference temperature:

\begin{eqnarray}
	\sum_r \nu_r h_{f,r} - \sum_p \nu_p h_{f,p} = Q_{react}
	\label{eq:enthalpy}
\end{eqnarray}

Considering the combustion of raw coal:

\begin{equation}
	\cee{C_xH_y O_z N_w + (x + y/4 - z/2) O2 -> x CO2 + y/2 H2O + w/2 N2}
	\label{eq:volatil}
\end{equation}

The raw coal molecule can be calculated in the following way:

\begin{eqnarray}
	\mu_i = y_i \frac{M_{raw}}{M_i}
	\label{eq:raw_stoic}
\end{eqnarray}

Where $\mu_i$ is respectively $x$, $y$, $z$ and $w$ if $i$ is C, H, O or N.
Considering reaction in Eq.~\eqref{eq:volatil}, the energy balance in Eq.\eqref{eq:enthalpy} can be
written:

\begin{eqnarray}
	Q_{react}=LHV_{raw}\cdot M_{daf} = h_{f,raw} + (x + y/4 - z/2) h_{f,\ox} -x h_{f,\cd} -y/2
	h_{f,\wat} - w/2 h_{f,\ny}
	\label{eq:raw_enth}
\end{eqnarray}

Where $h_{f,i}$ is the enthalpy of formation of species $i$. From Eq.~\eqref{eq:raw_enth} the
enthalpy of formation of raw coal can be obtained, given its molecular weight.

\subsection{TAR}

If tar is considered in volatile matter, generally it may be assumed that no heat is
produced/absorbed during pyrolysis:

\begin{eqnarray}
	\cee{C_x H_y O_z N_w -> \nu_{char}C_{(s)} + \nu_{tar} C_n H_m O_p + \sum_i \nu_i M_i}
	\label{eq:pyrolysis}
\end{eqnarray}

The stoichiometric coefficient of each species can be calculated from the volatile yield expressed
as mass fraction:

\begin{eqnarray}
	\nu_i = \frac{f_i M_{raw}}{M_i}
	\label{eq:stoich}
\end{eqnarray}

Applying Eq~.\eqref{eq:enthalpy} to reaction~\eqref{eq:pyrolysis}, assuming that $Q_react=0$, the
enthalpy of formation of tar is:

\begin{eqnarray}
	\nu_{tar} h_{f,tar} = h_{f,raw} - \nu_{char} h_{f,char} - \sum_i \nu_i h_{f,i}
	\label{eq:enth_tar}
\end{eqnarray}

\subsection{No TAR}

If only species, which enthalpy of formation is know, were defined, Eq.~\eqref{eq:enthalpy} can be
used to calculate the heat of pyrolysis:

\begin{eqnarray}
	- Q_{pyro} \cdot M_{raw} = h_{f,raw} - \nu_{char} h_{f,char} - \sum_i \nu_i h_{f,i}
	\label{eq:qpyro}
\end{eqnarray}

Where $Q_{pyro}$ is the heat of pyrolysis per unit of mass of daf. It is positive if heat is
required for breaking coal structure bounds. Generally, it is expressed in terms of volatile matter:

\begin{eqnarray}
	Q_{pyro}^{vm} = \frac{Q_{pyro}}{1-f_{char}} 
	\label{eq:qpyrovm}
\end{eqnarray}



		

% chapter chapter name (end)
% section section name (end)

% section xfasf (end)


% section sdgdg (end)


\printbibliography

%\begin{figure}[h]
%	\centering
%		\begin{tikzpicture}
%			\begin{axis}[
%				xlabel=$x$,
%				ylabel=$x^2$]
%				\addplot[color=red,mark=*] coordinates {
%				(2,-2.8559703)
%				(3,-3.5301677)
%				(4,-4.3050655)
%				(5,-5.1413136)
%				(6,-6.0322865)
%				(7,-6.9675052)
%				(8,-7.9377747)
%				};
%			\end{axis}
%		\end{tikzpicture}
%	\caption{dfadg}
%	\label{fig:prova}
%\end{figure}

\end{document}


